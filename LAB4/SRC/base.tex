\documentclass[12pt]{article}
\usepackage{pdfpages}
\usepackage{eso-pic}
\usepackage{hyperref,url}
\usepackage{graphicx}
\graphicspath{{images/}}
\newcommand\tab[1][1cm]{\hspace*{#1}}

\begin{document}
\input{Titlu.tex}
\cleardoublepage

\section*{Lucrarea de laborator \#1}
\phantomsection

\section{Scopul lucrarii de laborator}
De a se invata utilizarea unui Version Control System si modul de setare a unui server.
\section{Obiective}
Studierea Version Control Systems (git).

\clearpage

\cleardoublepage

\newpage
\section*{Concluzii}
\phantomsection

\tab In lucrarea data s-a creat o aplicatie mobila pe \textbf{IOS}.
Insusi aplicatia reprezinta o joaca (Tic-Tac-Toe). Joaca suporta doua regimuri.
Dupa fiecare joc cistigat - jucatorii acumuleaza puncte. Pe parcursul lucrarii s-a utilizat
masina virtuala pentru a virtualiza sistemul de operare Mac OS Sierra. Ca IDE s-a folosit
\textbf{XCode 8.3.1}. Au fost adaugate butoane de resetare a jocului, totodata dupa fiecare runda
cistigata - se evedentiaza cistigatorul. In urma efectuarii lucrarii am acumulat multa experienta pe
mobile, totodata am studiat sistema MAC si am invatat limbajul \textbf{Objective-C}.
\end{document}
