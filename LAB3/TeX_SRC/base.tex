\documentclass[12pt]{article}
\usepackage{pdfpages}
\usepackage{eso-pic}
\usepackage{hyperref,url}
\usepackage{graphicx}
\graphicspath{{images/}}
\newcommand\tab[1][1cm]{\hspace*{#1}}

\begin{document}
\input{Titlu.tex}
\cleardoublepage

\section*{Lucrarea de laborator \#1}
\phantomsection

\section{Scopul lucrarii de laborator}
De a se invata utilizarea unui Version Control System si modul de setare a unui server.
\section{Obiective}
Studierea Version Control Systems (git).

\clearpage

\cleardoublepage

\newpage
\section {Concluzii}
\tab In lucrarea data am invatat cum sa cream un site si sa lucram eficient
cu front-endul si back-endul. Este prima data cind folosesc \textbf{Python/Django}.
In urma lucrarii am acumulat multa experienta in web development. Este primul
meu pas in ceea ce priveste web-ul. Bazele de date au fost foarte usor de
manipulat prin Django. Python permite foarte usor sa lucram cu back-endul.
Pentru elemente de front-end am utilizat \textbf{HTML/CSS}. Am creat paginile
About, si totodata Images App. Plusul este ca putem crea mai multe baze de date, aplicatii. Situl poate fi extins. Asta si este plusul in Django. Dupa parerea
mea este un framework foarte puternic si este usor de inteles pentru un incepator.
\end{document}
